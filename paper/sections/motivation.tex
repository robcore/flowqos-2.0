\section{Motivation}
\label{sec:motivation}
Congestion has been, and continues to be, one of the most prevalent problems in networking.  One of the key issues with TCP is that when there are many flows it has a tendency to cause congestion, leading to degraded performance of competing flows.  This degradation can range in cost from minimal to disastrous.  When TCP connections consume all the available bandwidth real-time applications have a tendency to suffer the most. This is true because real-time applications have tighter latency and throughput constraints.

Our motivation for exploring FlowQoS is to provide better quality streaming services, which cannot afford to compete with flows like TCP which can dominate them.  It is commonplace for many users to have multiple web pages with active web traffic while simultaneously using a streaming service like Skype or VoIP- we believe that flow classification described in FlowQoS can prevent these users from receiving poor Skype call quality due to congestion that they may be creating.

One of the downsides of FlowQoS is that it statically allocates bandwidth to classes of traffic, such as TCP or UDP.  This is detrimental to throughput when certain classes are not fully utilizing their allotted bandwidth.  By using the concept of classifying traffic from FlowQoS and dynamically allocating bandwidth we will show that we can still achieve get the target quality of service that FlowQoS provides, while providing a total throughput that is closer to a non-regulated network.