\section{Motivation}
\label{sec:motivation}

%draft
Consider the case of physical network connection for a host that provides maximum total bandwidth B Kbps. Now, consider two different traffic types being forwarded through this link – say, video (UDP) traffic and file transfer (TCP) traffic. Let the video traffic be running at some bit-rate that produces an equivalent traffic at rate ‘x’ Kbps. Now the TCP traffic starts and as there is (B-x) Kbps of bandwidth still available, it increases it window size by maximum segment size MSS every RTT msecs. It linearly increases its rate until it tries to send traffic more than (B-x) Kbps, thereby increasing the total traffic exceeds the threshold of B Kbps that the link is capable of sending. As it is not possible to send traffic at that rate, both video and TCP traffic face packet losses. TCP based on its congestion control algorithm, halves its window size, and hence the rate. But, it again tries to linearly increase the rate and crosses the (B-x) Kbps available bandwidth threshold. This happens continuously until the video streaming service adapts itself to a lower bitrate and therefore a lower datarate ‘y’ (< ‘x’) Kbps. But, this only gives TCP more room to expand and repeating the same process over again, thereby reducing the bitrate even further, until an equilibrium is reached.\\

However, if we isolate the TCP and video traffic in separate queues/channels that put a hard rate limit on the either type of traffic, we can ensure that TCP can never try to consume the bandwidth required/used by the video traffic. FlowQoS helps us achieve the same effect, thereby providing the required Quality of Service to the sensitive traffic types.\\