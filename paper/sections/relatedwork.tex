\section{Related Works}
\label{sec:related-works}

% previous work for QoS
Quality of Service (QoS) is important in home networks that share a common bandwidth-limited Internet connection. Over the past several years, there has been a significant amount of research for QoS in IP networks~\cite{aurrecoechea1998survey, mcdysan1999qos, newman1996ipsilon}, traffic shapers\cite{Meraki}, and traffic flows classifiers~\cite{roughan2004class}. However, most previous approaches are different with FlowQoS either focusing on different issues or working in different scenarios. FlowQoS focuses in particular on making per-flow, application-based QoS, which is designed to deploy and configure in home networks. Most of QoS work 

Kim et al. provides a network QoS control framework~\cite{kim2010automated}, which sets rate limiters at the edge switches and priority queues for flow at each path hop. It uses a QoS control framework to manage automatically OpenFlow networks with multiple switches. On the contrary, FlowQoS provides similar automated trafiic shaping at a single gateway. Ishimori et al. introduces a novel QoSFlow framework to enhance QoS management procedures in OpenFlow networks~\cite{ishimori2012automatic}. QoSFlow shares similar shaping mechanism with FlowQoS. However, it does not focus on providing usable QoS for broadband access networks and is still under development. 
Ko et al. proposed a two-tier flow-based QoS management framework~\cite{nam2013openqflow}, which needs multicore processors and is not designed for home networks like FlowQoS. Ferguson et al. developed an API for applications to control a software-defined network (SDN). They call their prototype controller PANE~\cite{ferguson2013participatory}. PANE delegates read and write authority from the network’s administrators to end users, or applications and devices acting on their behalf. It allows users to work with the network, rather than around it. However, PANE addresses on more issues than FlowQos such as better performance, security and fairness etc., which makes it do not focuse on QoS in broadband access networks or application identification. Williams et al.~\cite{williams2011real} developed an automated IP traffic classification algorithm based on statistical flow properties. This approach limited the throughout of commodity home routers to 28 Mbps. In contrast, FlowQoS faces no such limitations.\\


%SDN-based solutions for home and broadband access networks

Current residential router QoS support must be manually configured and set up is difficult for the average home users. The emergence of OpenFlow/SDN is one of the most promising and disruptive networking technologies of recent years. It provides more possible solutions for QoS. In the SDN architecture, the control and data planes are decoupled. The underlying network infrastructure is abstracted from the application, which mitigate the issue that most users are not skilled network operators. There is also a lot of work based on SDN. Risso et al.~\cite{risso2012customizing} developed an OpenFlow-based mechanism for customizing data-plane processing in home routers, which makes it possible to install third parties' applications on the data plane of a router. The architecture is focused on more general data-plane modifications, not QoS. Georgopoulos et al.~\cite{georgopoulos2013towards} proposed an OpenFlow-assisted QoE fairness framework that aims to fairly improves users’ quality of experience (QoE) of multiple competing clients in a shared environment like home networks. However, the system performs per-device QoS instead of per-application or per-flow QoS. Mortier et al.~\cite{mortier2011supporting} designed and built a home networking platform named Homework to provide detailed per-flow measurement and management capabilities for home networks. However, Homework does not provide QoS support or application identification. 

%Previous work on HTB
